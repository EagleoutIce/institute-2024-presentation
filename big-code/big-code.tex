\documentclass{article}
% 10.5281/zenodo.4912025
% probably needs a larger tex memory to build :3
\usepackage[active,tightpage]{preview}
\makeatletter
\usepackage[encoding,cpalette,fakeminted,highlights]{../sopra-collection/sopra-listings/sopra-listings}
\appto\input@path{{../sopra-collection/sopra-listings/}}
\usepackage{xcolor}
\definecolor{uulmaccent}{HTML}{A9A28D}

\usepackage{../color-palettes/color-palettes}
\DefinePalette{RSlicing}
{Red,rötlich: named(uulmaccent)}
{Red,rötlich: named(uulmaccent)}
{Blau,bläulich: named(uulmaccent)}
{Lila,lilafarben: named(uulmaccent)}
% {Grün,grünlich: RGB(21, 150, 90)}
\SetShadeContrast{51}
\UsePalette{RSlicing}

\usepackage[margin=0mm,a4paper]{geometry}
\usepackage{amsmath}

\def\PreviewBorder{0mm}
\solsetmintedstyle{plain}
\solLoadLanguage{present-R}

\def\ColFrom{}
\def\ColTo{}
\def\HookColor{\gpreto\lsthk@OutputBox{\ColFrom\appto\lst@righthss{\ColTo}}}
\def\DoGray{\gdef\ColFrom{\color{gray!17}\mdseries}}
\def\DoColor{\gdef\ColFrom{\color{black}\bfseries}}
\let\DoReset\DoColor
\begin{document}
\preview\footnotesize
\begin{minted}{R}
#R version 4.0.5 (2021-03-31)
library("groundhog")
p <- c("tidyverse", "broom", "svglite", "gdtools", "GenSA", "gghighlight")
groundhog.library(p, "2021-02-15")
rm(p)

# latvina for cultivar names
sessionInfo(); Sys.setlocale("LC_ALL", "Latvian") # lv valoda

# set the data directory and load workspace
setwd("G:/Shared drives/Fenologija/Raksti/abeles/Zenodo/")
load("Apple_Pure_phenology_R_workspace_image.RData")

# -----------------------------------------
# scirpt to recreate components included in of "Apple_Pure_phenology_R_workspace_image.RData"
{
# --------------------------
# phenology data subset having at least 14 observations owerlaping wiht meteorology data set
{
   dM <- d %>%
      filter(Year %in% (meteoH$e_obs$Year))
   dM <- dM %>%
      group_by(Variety) %>%
      summarize(N = n()) %>%
      arrange(N) %>%
      filter(N >= 14) %>% # 12 skirnes
      select(-N) %>%
      left_join(dM)
}

# ----------------------
# FUNCTION definitinos
# phenology model deifned as functino, see Kalvans et al, 2015, DDcos model for details
{
   # parameter set for model testing
   pari = c(Tb = 10.1, DD = 200.1) # Tb - base temperature; DD - cumulative degree days for phase onset
   # lookup table for sin distribution
   sadalijums = round((sin(seq(0, 1, 0.01) *2 *pi) + 1)/2, 2)

   # elementary function, start of heat sum accumulation Januar 1
   DDsin_el <- function(pari,
                        meteoi,
                        sadalijums) {
      # pari - model parameters (Tb and DD)
      # meteoi - meteorology data
      # sadalijums - lookup table for sin distributino
      a <- meteoi %>%
      mutate(Tspread = Tmax - Tmin,
               DD = map2(Tspread, Tmin, ~(sadalijums * .x) + .y),
               DD = map(DD, ~(. -pari["Tb"])),
               DD = map_dbl(DD, ~replace(., . < 0, 0) %>% sum()),
               DD = DD / length(sadalijums),
               DD = cumsum(DD)) %>%
      filter(DD >= pari["DD"]) %>%
      filter(DoY == min(DoY)) %>%
      select(Date, DoY, DD) %>%
      rename(Analysis_DoY = DoY)
      return(a)
   }
   # elementary function test
   DDsin_el(pari = pari,
            meteoi = meteo$e_obs$data[[1]],
            sadalijums = sadalijums)

   # applyign the elementary functino to full data set
   DDsin <- function(pari,
                     sadalijums,
                     dd,
                     meteod) {
      # dd - phenology data,
      # meteod - meteorology data to be used
      # pari - phenology model paramters (Tb, DD)
      # Tb - bazes temepratura

      inner_join(dd, meteod, by = c("Year")) %>%
      filter(!is.na(data)) %>%
      mutate(Analysis_DoY = purrr::map(data, ~DDsin_el(pari = pari,
                                                         meteoi = .,
                                                         sadalijums = sadalijums))) %>%
      unnest(cols = c(Analysis_DoY))
   }

   # test the model for furll data set
   DDsin(pari = pari,
         sadalijums = sadalijums,
         dd = d[c(1:10),],
         meteod = meteo$e_obs)

   # calculating the model preformance indicators
   # (Nstat - number of data points; ME - mean erro; MAE - mean absolute error; RMSE - root mean squared error)
   DDsin_stat <- function(pari,
                           stats,
                           sadalijums,
                           dd,
                           meteod) {

      # pari - parameter set for phenology model (Tb, DD)
      # stats - to be returned, charatet vector, any combinatio of Nstat, ME, MAE, RMSE
      # sadalijums - sin distribution lookutable
      # dd - phenology data
      # meteod - meteorology data

      analyse <- DDsin(pari = pari,
                     sadalijums = sadalijums,
                     dd = dd,
                     meteod = meteod)
      statistics <- analyse %>%
      ungroup() %>%
      mutate(dif = Full_bloom_yd - Analysis_DoY) %>%
      summarize(Nstat = n(),
                  ME = mean(dif, na.rm = T),
                  MAE = mean(abs(dif), na.rm = T),
                  RMSE = sqrt(mean(dif^2, na.rm = T)))

      statistics %>%
      select(contains(stats)) %>%
      unlist()
   }
   DDsin_stat(pari = pari,
               stats = c("ME", "RMSE"),
               sadalijums = sadalijums,
               dd = d[c(1:20), ],
               meteod = meteo$e_obs)

}

# model optimizatino
{
   DDsinOpt <- function(pari, dd, meteod, maxit,
                        loweri, upperi) {

      # pari - phenology model initial paramters (Tb, DD)
      # dd - phenology data
      # meteod - meteorological data
      # maxit - maximum number of GenSA iterations, set to 1 for testing
      # loweri - lower bound of model parameter range  (Tb, DD)
      # upperi - upper bound of model parameter range  (Tb, DD)

      # if pari == NA a random set is generated
      if (any(is.na(pari))) {
      pari = runif(2) * (upperi - loweri) + loweri
      }

      # test if the phenology and meteorology data oweralap in time
      if (any(dd$Year %in% meteod$Year)) {
      print(Sys.time())

      print("DDsinOpt")
      a <- GenSA::GenSA(par = pari,
                        fn = DDsin_stat,
                        lower = loweri,
                        upper = upperi,
                        stats = c("RMSE"),
                        sadalijums = sadalijums,
                        dd = dd,
                        meteod = meteod,
                        control = list(maxit = maxit,
                                       verbose = T,
                                       temperature = 10000,
                                       smooth = T,
                                       simple.function = T))


      return(a$par)
      } else {
      print("meteoorology and phneology data do not owerlap!")
      c(Tb = NA, DD = NA)
      }
   }
   # test the optimizatino function
   DDsinOpt(pari = NA, dd = d[c(200:230),], meteod = meteoH$e_obs, maxit = 1,
            lower = c(Tb = 0, DD = 100), upper = c(Tb = 10, DD = 400))
   DDsinOpt(pari = c(Tb = 10.1, DD = 200.1), dd = d[c(220:235),], meteod = meteoH$e_obs, maxit = 2,
            lower = c(Tb = 0, DD = 100), upper = c(Tb = 11, DD = 400))

}

# -----------------------------------------
# MODEL OPTIMIZATION
# model optimization by cultivar and meteorology data source, to compare the model fit between data sources
{
   # emty object to store optimizatio results
   rez = c()
   # initial parameter set;
   pari = c(Tb = NA, DD = NA)
   for(i in c(1:100)) {
      print(i)

      di <- dM %>%
      group_by(Variety) %>%
      nest() %>%
      mutate(data = map(data, ~mutate(., Cal = as.logical(rbinom(dim(.)[1], size = 1, prob = 0.5)))),
               Ncal = map_dbl(data, ~summarize(., Ncal = sum(Cal)) %>% unlist()))

      for (n in names(meteoH)) {
      print(n)
      a <- di %>%
         mutate(Gensa = map(data, ~DDsinOpt(pari = pari, dd = filter(., Cal), meteod = meteoH[[n]], maxit = 5,
                                             lower = c(Tb = 0, DD = 100), upper = c(Tb = 11, DD = 400)) %>%
                              t() %>% as.data.frame()),
               Stat = map2(Gensa, data, ~DDsin_stat(pari = unlist(.x),
                                                      stats = c("Nstat" ,"ME", "MAE","RMSE"),
                                                      dd = .y %>% filter(!Cal),
                                                      sadalijums = sadalijums,
                                                      meteod = meteoH[[n]])  %>%
                              t() %>% as.data.frame())) %>%
         select(Variety, Ncal, Gensa, Stat) %>%
         unnest(cols = c(Gensa, Stat)) %>%
         mutate(Meteo = n)
      rez = bind_rows(rez, a)
      rm(i, a)

      # write_rds(rez, "G:/Shared drives/Fenologija/Raksti/abeles/rez_R/fit_5_fix_t0_100_MeteoTest_min14.rds")
      }
      rm(n,i)
   }
   # save the results
   write_rds(rez, "G:/Shared drives/Fenologija/Raksti/abeles/rez_R/fit_all.rds")
   # cehck the results
   tail(rez)
   meteo_comparison <- rez; rm(rez)
}

# model optimization with e-obs data set lumped all varieties
{
   # emty object to store optimizatio results
   rez = c()
   # initial parameter set;
   pari = c(Tb = NA, DD = NA)
   for(i in c(1:100)) {
      print(i)

      di <- d %>%
      group_by(Variety) %>%
      nest() %>%
      mutate(data = map(data, ~mutate(., Cal = as.logical(rbinom(dim(.)[1], size = 1, prob = 0.5)))),
               Ncal = map_dbl(data, ~summarize(., Ncal = sum(Cal)) %>% unlist()))


      a <- di %>%
      mutate(Gensa = map(data, ~DDsinOpt(pari = pari, dd = filter(., Cal), meteod = meteoH[[n]], maxit = 5,
                                          lower = c(Tb = 0, DD = 100), upper = c(Tb = 11, DD = 400)) %>%
                           t() %>% as.data.frame()),
               Stat = map2(Gensa, data, ~DDsin_stat(pari = unlist(.x),
                                                   stats = c("Nstat" ,"ME", "MAE","RMSE"),
                                                   dd = .y %>% filter(!Cal),
                                                   sadalijums = sadalijums,
                                                   meteod = meteo$e_obs)  %>%
                           t() %>% as.data.frame())) %>%
      select(Variety, Ncal, Gensa, Stat) %>%
      unnest(cols = c(Gensa, Stat)) %>%
      mutate(Meteo = n)
      rez = bind_rows(rez, a)
      rm(i, a)

      }
   rm(n,i)
   # save the results
   write_rds(rez, "G:/Shared drives/Fenologija/Raksti/abeles/rez_R/e_obs_noVariety.rds")
   # cehck the results
   tail(rez)
   e_obs_noVariety <- rez; rm(rez)
}
}

# -----------------------------------------
# ILUSTRATIONS
# Figure 1 periodu medians
{
   dMed_year <- d %>%
      group_by(Year) %>%
      summarize(Med = median(Full_bloom_yd))

   dMed_periods <-
      tibble(Per = c("I", "II", "III"), From = c(1959, 1976, 2002), To = c(1967, 1987, 2019)) %>%
      mutate(Med = map2_dbl(From, To, ~filter(d, Year >= .x, Year <= .y) %>%
                              summarize(Med = median(Full_bloom_yd)) %>% pull())) %>%
      gather(Par, Value, -Per, -Med)
   print(dMed_periods)

   a <- d %>%
      ggplot(aes(Year, Full_bloom_yd)) +
      geom_jitter(shape = 1, width = 0.2, height = 0) +
      geom_line(data = dMed_periods, mapping = aes(Value, Med, group = Per), lwd = 2, alpha = 0.5) +
      scale_x_continuous(minor_breaks = c(1957, 1958, 1959:2019, 2020, 2021),
                        breaks = seq(1960, 2015, 5)) +
      ylab("Full flowering, day of the year")
   print(a)

   rm(dMed_year, dMed_periods, a)
}

# Figure 2 meteo salidzinajums
{
   a <- meteo_salidzinajums %>%
      mutate(Meteo = replace(Meteo, Meteo == "meteo.lv", "Stende meteorological station")) %>%
      select(-Ncal, -Nstat, -Tb, -DD) %>%
      mutate(Variety = str_replace(Variety, " ", "\n")) %>%
      gather(Error, Value_e, -Variety, -Meteo) %>%
      ggplot(aes(Variety, Value_e, fill = Meteo)) +
      geom_boxplot() +
      facet_wrap(~Error, scales = "free") +
      ylab("Model error, days") +
      labs(fill = 'Meteorological data source') +
      theme(axis.text.x = element_text(angle = 90, vjust = 0.5, hjust=1),
            legend.position = "bottom")
   print(a)
   rm(a)

   # statistika
   meteo_salidzinajums %>%
      select(-Ncal, -Nstat, -Tb, -DD) %>%
      gather(Error, Value_e, -Variety, -Meteo) %>%
      group_by(Error, Meteo) %>%
      summarize(Med = median(Value_e)) %>%
      as.data.frame()
}

# Figure 3
# Tb un DD e-obsam, visual inspection
{

   a <- e_obs_noVariety %>%
      ggplot(aes(Tb, DD)) +
      geom_point(shape = 1)
   a <- ggplotGrob(a)
   b <- e_obs_noVariety %>%
      ggplot(aes(Tb, RMSE)) +
      geom_point(shape = 1)
   b <- ggplotGrob(b)
   cc <- e_obs_noVariety %>%
      ggplot(aes(RMSE, DD)) +
      geom_point(shape = 1)
   cc <- ggplotGrob(cc)
   blank <- ggplot() + theme_void()
   blank <- ggplotGrob(blank)


   abc <- gridExtra::arrangeGrob(cbind(rbind(a, b, size = "first"),
                                       rbind(cc, blank, size = "first"),
                                       size = "first"))
   grid::grid.newpage()
   grid::grid.draw(abc)

   rm(a, b, cc, abc, blank)
}

# a single best paramter (Tb, DD) set
{
   e_obs_noVariety_global <-
      e_obs_noVariety %>%
      rename(Nstat_l = Nstat,ME_l = ME, MAE_l = MAE,  RMSE_l = RMSE) %>%
      mutate(stats = map2(Tb, DD, ~ DDsin_stat(pari = c(Tb = .x, DD = .y),
                                             stats = c("Nstat", "ME", "MAE", "RMSE"),
                                             sadalijums = sadalijums,
                                             dd = d,
                                             meteod = meteo$e_obs)))

   e_obs_noVariety_global %>%
      mutate(stats = map(stats, ~array(., c(1, length(.)), dimnames = list(NULL, names(.))) %>% as.data.frame())) %>%
      unnest(cols = c(stats)) %>%
      arrange(RMSE) %>% as.data.frame()

   z <- e_obs_noVariety %>%
      filter(Tb >= 2.5, Tb <= 4.0) %>%
      arrange(RMSE)

   as.data.frame(z)
   range(z$DD)
}
rm(z, e_obs_noVariety_global)
\end{minted}
\endpreview
\end{document}