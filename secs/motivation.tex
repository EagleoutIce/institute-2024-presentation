% wir starten mit dem was uns jetzt wahrscheinlich alle fehlt
\Sec{Let's get some}{Motivation}
% \begin{frame}{~}
% \solSetStyle{keywordA}{}%
% \begin{focus}
%    % fragen ob die befehle allen so grob bekannt sind
% \begin{minipage}{.65\linewidth}
% Probleme mit Reproduzier- und Replizierbarkeit
% \end{minipage}%
% \end{focus}
% % stern danach ja nein einfach erklären
% \end{frame}

\def\LinkMark#1{\textcolor{gray}{\scriptsize#1}}
%\def\Rect#1#2{\draw[rounded corners,gray] ([shift={(-.5mm,.5mm)}]#1.north west) rectangle ([shift={(.5mm,-.5mm)}]#2.south east);}
\newcommand\BraceRight[4][2mm]{\draw[line cap=round, gray,decorate,decoration={brace}] ([shift={(#1,.5mm)}]#2.north west-|#3.east) to[edge node={node[right=1mm,font=\small,align=left] {#4}}] ([shift={(#1,-.5mm)}]#3.south east);}
\newcommand\BraceBelow[4][.5mm]{\draw[line cap=round, gray,decorate,decoration={brace,mirror}] ([shift={(-.5mm,-#1)}]#2.south west) to[edge node={node[below=1mm,font=\small] {#4}}] ([shift={(.5mm,-#1)}]#3.south east);}
\def\Arrow#1#2{\draw[Kite-,line cap=round,gray,font=\small] ([xshift=2mm]#1.east) -- ++(4.5mm,0) node[right=1mm] {#2};}
\def\ArrowBelow#1#2#3{\path (#1.south west)--(#2.south east) coordinate[pos=.5] (@); \draw[Kite-,line cap=round,gray,font=\small] (@) -- ++(0,-3mm) node[below] {#3};}
\def\Out#1{{\color{lightgray}\sout{#1}}}

% ToDo: https://cran.r-project.org/web/packages/groundhog/index.html
% could be used to check if r packages are rpeproducible and use the version

\colorlet{@highlight}{black!70!white!95!red}

\def\FileSpy#1{\scope[spy using outlines={rectangle, magnification=10,width=8cm,height=4cm,rounded corners=1pt,line cap=round,connect spies}]#1\endscope}

\def\FileHeight{6cm}
\def\ShowFile{\node[fill=white] (@) at (0,0) {\includegraphics[height=\FileHeight]{./big-code/build/big-code.pdf}};}
\def\FileCite{\begin{tikzpicture}[@O]
   \node[FCite] at(current page.south west) {\bottomcite{drudze_2021_4912025}};
\end{tikzpicture}}
\tikzset{file spy goal/.style={
   rounded corners=4pt,
   drop shadow={fill=lightgray!50}
}}
\def\FileBasis#1#2#3#4{
\begin{tikzpicture}
\FileSpy{
   \onslide<#1>{\ShowFile}
   \only<#2>{\spy[opacity=.9] on ([yshift=-4.5mm,xshift=-1.33mm]@.north) in node (@b) [file spy goal] at (5,0);}
   #4
}
   #3
\end{tikzpicture}\onslide<#1>{\FileCite}
}

% needs a larger tex memory to build :3
% 1-Zenodo_Output.csv-output/695/2_Apple_Pure_phenology_R_script.r

\def\Scripts{\textcolor{lightgray}{R Scripts}\space}

\begin{frame}[c]{\Scripts Fail to Replicate}
\begin{layout-full}
\FileBasis{1-}{1-}{% 371 lines (avg. ~348)
   \only<2->{
      \draw[thick,@highlight,rounded corners=2pt] ([xshift=-2.5cm,yshift=-6.1mm]@b.north east) node[below left,fill=white,fill opacity=.5,text opacity=1] {{\bfseries\tiny Hardcoded Paths}};
   }
   \only<4->{
      \draw[thick,@highlight,rounded corners=2pt]  ([xshift=-2.05cm,yshift=-9.25mm]@c.north east) node[below left,fill=white,fill opacity=.5,text opacity=1] {{\bfseries\tiny (Pseudo-)Randomness without Seed}};
   }
}{
   \only<3->{
      \spy[opacity=.9,height=3cm] on ([yshift=-24.5mm,xshift=-1.33mm]@.north) in node (@c) [file spy goal] at (5.33,-1.1);
   }
}
\end{layout-full}
\begin{tikzpicture}[@O]
   \onslide<5->{
      \node[below left=6mm,yshift=-5mm] at(current page.north east) {%
         \textbf{\qty{74}{\percent} even fail to complete!}\supercite{trisovic_largescale_2022}%
      };
      \node[FCite] at(current page.south west) {\bottomcite{trisovic_largescale_2022}\\\strut};
   }
\end{tikzpicture}
\note[itemize]{
   \item hardcoded paths, scattered througout the script
   \item randomness without seed, unknown if intendent or differences are due to randomness
   \item up to you the reviewer to decide if differences in the results are intendet, due to randomness or errors
   \item problems backed up by a study from Trisovic with R projects as parts of publications: 74\% fail to run to completion complete
   \item missing libraries or files
   \item syntax errors
   \item missing setup
   \item **Next Slide**
}
\end{frame}

   % 371 lines divided by \FileHeight to get per-line height we try to fake pad it a little bit
   \pgfmathsetmacro{\LineHeight}{\FileHeight/372}
\def\AtLine#1#2#3{%
   \def\tmp{#2}%
   \ifx\tmp\empty
   \path[]
      ([yshift=-\LineHeight pt*#1+3.85mm]@.north east) coordinate (s) to[edge node={node[right=1.25mm] {\small#2}}]([yshift=-\LineHeight pt*#3+3.8mm]@.north east) coordinate (e);
   \else
   \draw[decoration={brace},decorate,@highlight,thick,line cap=round]
      ([yshift=-\LineHeight pt*#1+3.85mm]@.north east) coordinate (s) to[edge node={node[right=1.25mm] {\small#2}}]([yshift=-\LineHeight pt*#3+3.8mm]@.north east) coordinate (e);
   \fi
   \fill[@highlight,opacity=.15,rounded corners=.65pt] ([xshift=-1mm]s) rectangle ([xshift=1mm]e-|@.south west);

}

\begin{frame}{The Reproducibility Crisis}
   \begin{itemize}
      \itemsep8pt
      \item<2-> We have a problem with reproducibility\supercite{baker20161}
      \item<3-> Consequently, with FAIR research\supercite{barker2022introducing} \begin{itemize}
         \item Findable % (DOI, in registry, ....)
         \item Accessible % (open access, ...)
         \item Interoperable % (standards, ...)
         \item Reusable % (license, executable, ...); hier hören wir auch von anderen (Jakob mit Zenodo usw.)
      \end{itemize}
   \end{itemize}
   \begin{tikzpicture}[@O]
      \node[FCite] at(current page.south west) {%
      \bottomcite{baker20161}\\
      \bottomcite{barker2022introducing}
      };
   \end{tikzpicture}
\end{frame}
% \Learning{Use Macros to Ease Your Workflow!}{At leat some, e.g., for names, to reduce repetition\ldots}
% # structured texing